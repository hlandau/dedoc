% vim: fdm=marker fdl=0
\usemodule[mathml]

%% XML TRANSLATION                                                         {{{1
%%%%%%%%%%%%%%%%%%%%%%%%%%%%%%%%%%%%%%%%%%%%%%%%%%%%%%%%%%%%%%%%%%%%%%%%%%%%%%%
\xmlregisterdocumentsetup{dedoc}{xs:base}
\xmlregisterns{s}{https://www.devever.net/ns/dedoc}

\startxmlsetups xs:base
  \xmlsetsetup{#1}{s:doc|s:docbody|s:sec|s:hdr|s:p|s:ul|s:ol|s:li|s:dict|s:dice|s:dick|s:dicb|s:listing|s:tex|s:em|s:tt|s:footnote|s:proword|s:procn|s:kw}{xs:*}
  \xmlsetsetup{#1}{s:sec}{xs:sec1}
  \xmlsetsetup{#1}{s:sec/s:sec}{xs:sec2}
  \xmlsetsetup{#1}{s:sec/s:sec/s:sec}{xs:sec3}
  \xmlsetsetup{#1}{s:sec/s:sec/s:sec/s:sec}{xs:sec4}
  \xmlsetsetup{#1}{s:sec/s:sec/s:sec/s:sec/s:sec}{xs:sec5}
  \xmlsetsetup{#1}{s:sec/s:sec/s:sec/s:sec/s:sec/s:sec}{xs:sec6}
\stopxmlsetups

\startxmlsetups xs:doc
  \xdef\DocTitle{\xmltext{#1}{/s:docctl/s:title}}
  \xdef\DocRevSummaryLong{\xmltext{#1}{/s:docctl/s:buildinfo/s:vcsrevsummary[@form='long']}}
  \xdef\DocRevSummaryShort{\xmltext{#1}{/s:docctl/s:buildinfo/s:vcsrevsummary[@form='short']}}
  \xdef\DocRev{\xmltext{#1}{/s:docctl/s:buildinfo/s:vcsrev}}
  \xdef\DocTime{\xmltext{#1}{/s:docctl/s:buildinfo/s:vcstime}}
  \setvariables[document][title={\DocTitle}]

  % Set PDF metadata.
  \setupinteraction[title={\DocTitle},author={Hugo Landau}]
  \setupinteractionscreen[option=bookmark]
  \placebookmarks[chapter,section,subsection,subsubsection]

  %% Cover Page                                                             {{{2
  %%%%%%%%%%%%%%%%%%%%%%%%%%%%%%%%%%%%%%%%%%%%%%%%%%%%%%%%%%%%%%%%%%%%%%%%%%%%%%
  \noheaderandfooterlines
  \begingroup
    %% context-xml changes the catcodes so that $ doesn't do anything for XML
    %% processing-related reasons. TikZ complains about this, so we need to
    %% temporarily restore it.
    \pushcatcodetable
    \setcatcodetable\ctxcatcodes

    %% The cover page uses a background layer; the page itself is an empty page.
    \definelayer[cover][x=0mm,y=0mm,width=\paperwidth,height=\paperheight]
    \setlayer[cover]{%
    \starttikzpicture[overlay]
      %% Fill white page with white. Ensure TikZ desn't relativise our coordinates.
      \fill[fill=white] (0,0) rectangle (210mm,-297mm);

      %% Document title.
      \draw[anchor=west] (5\bodyfontsize,-100mm) node {\bfd\DocTitle};

      %% The MAIN STRIKE insignia.
      \startscope[xshift=150.1mm,yshift=-281mm]
       \startscope[scale=0.060]
        %\draw (0,0) rectangle (1000mm,1000mm);
        \fill (863.200mm,596.700mm) -- (0,235.900mm) -- (0,0);
        \fill (863.200mm,596.700mm) -- (520.600mm,783.400mm) -- (522.600mm,750.400mm);
        %\fill[fill=black] (0,0) rectangle (1000mm,1000mm);
        %\fill[fill=white] (863.200mm,596.700mm) -- (0,235.900mm) -- (0,0);
        %\fill[fill=white] (863.200mm,596.700mm) -- (520.600mm,783.400mm) -- (522.600mm,750.400mm); 
       \stopscope
      \stopscope

      %% Document control information.
      \draw[anchor=south west,align=left] (5\bodyfontsize,-281mm) node[fill=white,draw=black,inner sep=5mm] {{\tt //devever.net/...}
        \doifelse{\DocRevSummaryShort}{}{}{\\ \DocRevSummaryShort}%
      };
    \stoptikzpicture%
    }

    \setupbackgrounds[page][background=cover]
    \popcatcodetable

    %% The empty page.
    \startstandardmakeup[pagestate=start] % Ensure page is numbered
    \stopstandardmakeup
  \endgroup

  %% Copyright Page                                                         {{{2
  %%%%%%%%%%%%%%%%%%%%%%%%%%%%%%%%%%%%%%%%%%%%%%%%%%%%%%%%%%%%%%%%%%%%%%%%%%%%%%
  \begingroup
    \startstandardmakeup[pagestate=start] % Ensure page is numbered
    \vfill
    %\xmlfirst{#1}{/s:docctl/s:docinfo/s:dedication}
    \vfill
    Feedback, errata: {\tt hlandau@devever.net}

    \vskip10mm

    {\tt //devever.net/...}\endgraf

    \doifelse{\DocRevSummaryLong}{}{}{\\ {\tt \DocRevSummaryLong} (\cldcontext{string.sub([[\DocTime]],1,10)})}%
    \vskip5mm

    System \ConTeXt{} · \DEDOC{}
    \vskip5mm

    © \doifelse{\DocTime}{}{}{\cldcontext{string.sub([[\DocTime]],1,4)}} Hugo Landau

    \stopstandardmakeup
  \endgroup

  \title{Table of Contents}
  \placecontent

  \xmlfirst{#1}{/s:docbody}
\stopxmlsetups
  %% }}}2

\startxmlsetups xs:docbody
\xmlflush{#1}%
\stopxmlsetups

%% DO NOT put \reference outside of a margin note. Placed after the heading
%% command, it causes a single space to appear before the first paragraph. Placed
%% before it, it randomly causes an empty line to appear to appear above the
%% heading. I spent an entire day debugging this.
\startxmlsetups xs:sec1
\chapter[id:\xmlatt{#1}{id}]{\xmlfirst{#1}{/s:hdr/s:title}}%
\inleftmargin{\reference[idpfx:\xmlattribute{#1}{.}{id}]{§ }}%
\xmlflush{#1}%
\stopxmlsetups

\startxmlsetups xs:sec2
\section[id:\xmlatt{#1}{id}]{\xmlfirst{#1}{/s:hdr/s:title}}%
\inleftmargin{\reference[idpfx:\xmlattribute{#1}{.}{id}]{§ }}%
\xmlflush{#1}%
\stopxmlsetups

\startxmlsetups xs:sec3
\subsection[id:\xmlatt{#1}{id}]{\xmlfirst{#1}{/s:hdr/s:title}}%
\inleftmargin{\reference[idpfx:\xmlattribute{#1}{.}{id}]{§ }}%
\xmlflush{#1}%
\stopxmlsetups

\startxmlsetups xs:sec4
\subsubsection[id:\xmlatt{#1}{id}]{\xmlfirst{#1}{/s:hdr/s:title}}%
\inleftmargin{\reference[idpfx:\xmlattribute{#1}{.}{id}]{§ }}%
\xmlflush{#1}%
\stopxmlsetups

\startxmlsetups xs:sec5
\subsubsubsection[id:\xmlatt{#1}{id}]{\xmlfirst{#1}{/s:hdr/s:title}}%
\inleftmargin{\reference[idpfx:\xmlattribute{#1}{.}{id}]{§ }}%
\xmlflush{#1}%
\stopxmlsetups

\startxmlsetups xs:sec6
\subsubsubsubsection[id:\xmlatt{#1}{id}]{\xmlfirst{#1}{/s:hdr/s:title}}%
\inleftmargin{\reference[idpfx:\xmlattribute{#1}{.}{id}]{§ }}%
\xmlflush{#1}%
\stopxmlsetups

\startxmlsetups xs:hdr
\stopxmlsetups

\startxmlsetups xs:p
\xmlflush{#1}\par%
\stopxmlsetups

\startluacode
function document.addfunnyhyphen(tfmdata)
  local underscore = utf.byte("_")
  local char = tfmdata.characters[underscore]
  if not char then return end
  tfmdata.characters[0xFE000] = {
    width = 0,
    height = 0,
    depth = 0,
    commands = {
      { "right", -char.width },
      {  "down", char.depth },
      { "slot", 1, underscore },
    }
  }
end

--utilities.sequencers.appendaction("aftercopyingcharacters", "after", "document.addfunnyhyphen")

local shared = {
  start = 1,
  length = 1,
  --before = utf.char(0xFE000),
  after = nil,
  left = false,
  right = false,
}

local all = table.setmetatableindex({}, function(t, k)
  return shared
end)

languages.hyphenators.traditional.installmethod("foo", function(dictionary, word, n)
  return all
end)

function xml.functions.processListing(t)
  buffers.assign("listingBuf", "\\begingroup\\starttyping[lines=hyphenated]\n" .. tostring(xml.text(t)) .. "\n\\stoptyping\\endgroup")
  context.getbuffer { "listingBuf" }
end
\stopluacode
%\definehyphenationfeatures[foo][characters=all,alternative=foo,righthyphenchar="FE000]

\startxmlsetups xs:listing
\pushcatcodetable
\setcatcodetable\ctxcatcodes
\xmlfunction{#1}{processListing}
\popcatcodetable
%{\tt \xmlflush{#1}}\par%
%{\tt \xmlflushspacewise{#1}}\par%
\stopxmlsetups

\startxmlsetups xs:ul
\startitemize%
\xmlflush{#1}%
\stopitemize
\stopxmlsetups

\startxmlsetups xs:ul
\startitemize[n]%
\xmlflush{#1}%
\stopitemize
\stopxmlsetups

\startxmlsetups xs:li
\startitem \xmlflush{#1} \stopitem%
\stopxmlsetups

\startxmlsetups xs:dict
\xmlflush{#1}%
\stopxmlsetups

\startxmlsetups xs:dice
\startdesc{\xmlfirst{#1}{/s:dick}}%
\xmlfirst{#1}{/s:dicb}%
\stopdesc%
\stopxmlsetups

\startxmlsetups xs:dick
\xmlflush{#1}%
\stopxmlsetups

\startxmlsetups xs:dicb
\xmlflush{#1}%
\stopxmlsetups

\startxmlsetups xs:tex
\xmlflushcoontext{#1}%
\stopxmlsetups

\startxmlsetups xs:em
  {\it \xmlflush{#1}}
\stopxmlsetups

\startxmlsetups xs:tt
  {\tt \xmlflush{#1}}
\stopxmlsetups

\startxmlsetups xs:footnote
\footnote{\xmlflush{#1}}%
\stopxmlsetups

\startxmlsetups xs:proword
  {\bf \xmlflush{#1}}
\stopxmlsetups

\startxmlsetups xs:procn
  {\sc \xmlflush{#1}}
\stopxmlsetups

\startxmlsetups xs:kw
  {\tt \xmlflush{#1}}
\stopxmlsetups

%math (inline)
%  tex
%  <mml:math/>

%term
%link
%cite


%% CONFIGURATION                                                           {{{1
%%%%%%%%%%%%%%%%%%%%%%%%%%%%%%%%%%%%%%%%%%%%%%%%%%%%%%%%%%%%%%%%%%%%%%%%%%%%%%%

% Font Fallback Hacks {{{2
% -------------------
%
% TeX remains in the stone ages as far as font fallback support is concerned.
% We have to manually list fallback ranges. This list of ranges is incomplete,
% so this is terribly broken.
\definefontfallback[cjk][name:ArialUnicodeMS][cjkunifiedideographs,cjksymbolsandpunctuation,cjkcompatibility,cjkunifiedideographsextensiona,hiragana,katakana,halfwidthandfullwidthforms,cyrillic]

% Use Helvetica (actually, TeX Gyre Heros) {{{2
% ----------------------------------------
%
% For the "1970s computer manual" look.
\definefontsynonym[Sans][heros][fallbacks=cjk]
\loadtypescriptfile[texgyre]

% Helvetica Math Hacks {{{2
% --------------------
%
% Helvetica for text _and_ math; but use LM not cursor for monoo,
% since cursor is a bit light.
\starttypescriptcollection[myhvmath]
  \starttypescript[myhvmath]
    \definetypeface[myhvmath][ss][sans][heros][default][rscale=1]
    \definetypeface[myhvmath][rm][serif][termes][default]
    \definetypeface[myhvmath][tt][mono][modern][default][rscale=1.05]
    \definetypeface[myhvmath][mm][math][modern][default][rscale=1]
  \stoptypescript
\stoptypescriptcollection

\setupbodyfont[myhvmath,ss,12pt]

% Paper Size {{{2
% ----------
\definepapersize[main][A4]
\definepapersize[lmain][A4,landscape]
\setuppapersize[main]

% Layout and Margins {{{2
% ------------------
\setuplayout[
  width=middle,
  height=middle,
  footer=3\bodyfontsize,
  header=2\bodyfontsize,
  headerdistance=\bodyfontsize,
  topspace=0.5\bodyfontsize,
  leftmargin=5\bodyfontsize,
  rightmargin=5\bodyfontsize]
\definelayout[lmain][
  width=\paperwidth,
  height=\paperheight,
  backspace=0mm]
\setupwhitespace[small]

% Structure and Heading Configuration {{{2
% -----------------------------------
\setupstructure[state=start,method=auto]
\definestructureconversionset[myconv][A,n,n,n,n,n,n,n][n]

% Make chapter/section headings have the section number in the margin.
\unexpanded\def\PlaceSection#1#2%
  {\goodbreak
   \vbox
     {\localheadsetup
      \begstrut
      \inleftmargin{ #1}%
      #2}}

\setupheads[part,chapter,section,subsection,subsubsection,
            subsubsubsection,subsubsubsubsection][sectionconversionset=myconv]
\setuphead[chapter][sectionsegments=chapter,page=yes,
                    before={\blank[10*big,force]},after=,
                    style=\bfc,command=\PlaceSection]
\setuphead[section][sectionsegments=chapter:subsubsubsubsection,style=\bf,
                    before=,after=,margin=0mm,distance=0mm,textdistance=0mm,
                    command=\PlaceSection,
                    interlinespace=off,alternative=normal]

%\setupitemgroup[itemize][left=\tfc,inbetween=]
\definedescription[desc][headstyle=bold,style=normal,align=flushleft,
                         alternative=hanging,width=broad,margin=1cm]

\defineresetset[default][1,0,1][1]
\setuphead[sectionresetset=default]
\setupcaptions[location=top,prefix=yes,way=bychapter,prefixsegments=chapter]%way=bysection

% Referencing {{{2
% -----------
\setupreferencing[left={},right={}]
\usebtxdefinitions[apa]

% Header Text {{{2
% -----------
\setupheadertexts[]
\setupheadertexts[\setups{text a}][][][\setups{text b}]

\startsetups[text a]
  \framed[width=\textwidth,frame=off,bottomframe=on]{\rlap{Page \pagenumber{} of \totalnumberofpages}
  \hfill
  {\sc \getmarking[chapter]}
  \hfill
  \llap{\sc \getvariable{document}{title}}{}}
\stopsetups

\startsetups[text b]
  \framed[width=\textwidth,frame=off,bottomframe=on]{\rlap{\sc \getvariable{document}{title}}
  \hfill
  {\sc \getmarking[chapter]}
  \hfill
  \llap{Page \pagenumber{} of \totalnumberofpages}}
\stopsetups

% Interaction {{{2
% -----------
\setupinteraction[state=start,color=,contrastcolor=,style=normal]

% Colors {{{2
% ------
\setupcolors[state=start]

% TikZ {{{2
% ----
\usemodule[tikz]
\usetikzlibrary{positioning,shapes.misc,arrows,arrows.meta,patterns,graphs,fit,calc}

% Verbatim Listings  {{{2
% -----------------

% 'Return' symbol for end of verbatim listings.
\startreusableMPgraphic{return}
  drawarrow
    (0,0) --
    (1EmWidth, 0) --
    (1EmWidth, -.5ExHeight) --
    (.5EmWidth, -.5ExHeight);
\stopreusableMPgraphic
\definesymbol[return][\reuseMPgraphic{return}]
\def\vcrlf{\symbol[return]\crlf\strut\kern1em\strut\ignorespaces}

% DEDOC Type {{{2
% ----------
\def\DEDOC{DEDOC}


%% MAIN                                                                    {{{1
%%%%%%%%%%%%%%%%%%%%%%%%%%%%%%%%%%%%%%%%%%%%%%%%%%%%%%%%%%%%%%%%%%%%%%%%%%%%%%%
\starttext
\xmlprocessfile{dedoc}{\ArgInput}{}
%\startbackmatter
%\stopbackmatter
\stoptext
