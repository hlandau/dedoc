\usemodule[mathml]
\xmlregisterdocumentsetup{dedoc}{xs:base}
\xmlregisterns{s}{https://www.devever.net/ns/dedoc}
\xmlregisterns{bib}{https://www.devever.net/ns/bib}


%% XML TRANSLATION                                                         {{{1
%%%%%%%%%%%%%%%%%%%%%%%%%%%%%%%%%%%%%%%%%%%%%%%%%%%%%%%%%%%%%%%%%%%%%%%%%%%%%%%
\startxmlsetups xs:base
  %\xmlsetsetup{#1}{*}{-}
  \xmlsetsetup{#1}{s:top|s:doc|s:docctl|s:docinfo|s:docbody|s:doccontent|s:docproper|s:docmain|s:docannex|s:clause|s:annex|s:p|s:title|s:code|s:tt|s:figure|s:tex|s:math2|s:pre|s:footnote|s:iref|s:ul|s:ol|s:li|s:hdr|s:termdef|s:em|s:procn|s:proword|s:term|s:table|s:fig-output|s:cite|xs:bitrange|xs:field|xs:reg|xs:regblock|s:part|s:dedication|s:tp|s:dict|s:dice|s:dick|s:dicb|s:bq|s:typeref|s:procref|s:span|s:tabular|s:tr|s:td|s:th|s:table|s:typepage|s:tpsynopsys|s:tpdescription|s:tpalso|s:description|s:dualcolumn|s:procpage|s:ppsynopsys|s:ppdescription|s:ppargs|s:ppprecon|s:ppalso|s:unchecked|s:pparg|s:ppword|s:argref|s:pppostcon|s:conlist|s:ppret|s:ppworld|s:pparglist|s:inlpfx|s:listing}{xs:*}
  \xmlsetsetup{#1}{s:fig}{xs:figother}
  \xmlsetsetup{#1}{s:fig[@type="tex"]}{xs:figtex}

  \xmlsetsetup{#1}{s:clause/s:clause}{xs:clause2}
  \xmlsetsetup{#1}{s:clause/s:clause/s:clause}{xs:clause3}
  \xmlsetsetup{#1}{s:clause/s:clause/s:clause/s:clause}{xs:clause4}
  \xmlsetsetup{#1}{s:clause/s:clause/s:clause/s:clause/s:clause}{xs:clause5}
  \xmlsetsetup{#1}{s:clause/s:clause/s:clause/s:clause/s:clause/s:clause}{xs:clause6}

  \xmlsetsetup{#1}{s:annex/s:clause}{xs:clause2}
  \xmlsetsetup{#1}{s:annex/s:clause/s:clause}{xs:clause3}
  \xmlsetsetup{#1}{s:annex/s:clause/s:clause/s:clause}{xs:clause4}
  \xmlsetsetup{#1}{s:annex/s:clause/s:clause/s:clause/s:clause}{xs:clause5}
  \xmlsetsetup{#1}{s:annex/s:clause/s:clause/s:clause/s:clause/s:clause}{xs:clause6}

  \xmlsetsetup{#1}{s:dict[@kind="also-grouped"]//s:dice}{xs:dicealsogrouped}
  \xmlsetsetup{#1}{s:dict[@kind="also"]}{xs:dictalso}
  \xmlsetsetup{#1}{s:dict[@kind="also"]//s:dice}{xs:dicealso}
  \xmlsetsetup{#1}{s:tpalso//s:ul}{xs:ulalso}

  \xmlsetsetup{#1}{bib:set|bib:entry}{xbib:*}

  %\xmlfilter{#1}{omt:*/function(remapopenmath)}
  %\xmlfilter{#1}{mml:bind/function(remapmmlbind)}
  %\xmlfilter{#1}{mml:csymbol/function(remapmmlcsymbol)}
  %\xmlsetsetup{#1}{mml:*}{mml:*}
  %\xmlsetsetup{#1}{mml:apply/mml:apply/mml:inverse/../..}{mml:apply:inverse}
  %\xmlstrip{#1}{(mml:mi|mml:mo|mml:mn|mml:csymbol)}
  %\xmlstripanywhere{#1}{!pre}
\stopxmlsetups

\startxmlsetups xs:top
  \xmlflush{#1}
\stopxmlsetups

\startxmlsetups xs:doc
  \xdef\DocTitle{\xmltext{#1}{/s:docctl/s:docinfo/s:doctitle}}
  \xdef\DocRevision{\xmlatt{#1}{revision}}
  \setvariables[document][title={\DocTitle}]

  % Set PDF metadata.
  \setupinteraction[title={\DocTitle},author={Hugo Landau}]
  \setupinteractionscreen[option=bookmark]
  \placebookmarks[chapter,section,subsection,subsubsection]

  %% Cover Page                                                             {{{1
  %%%%%%%%%%%%%%%%%%%%%%%%%%%%%%%%%%%%%%%%%%%%%%%%%%%%%%%%%%%%%%%%%%%%%%%%%%%%%%
  \noheaderandfooterlines
  \begingroup
    %% context-xml changes the catcodes so that $ doesn't do anything for XML
    %% processing-related reasons. TikZ complains about this, so we need to
    %% temporarily restore it.
    \pushcatcodetable
    \setcatcodetable\ctxcatcodes

    %% The cover page uses a background layer; the page itself is an empty page.
    \definelayer[cover][x=0mm,y=0mm,width=\paperwidth,height=\paperheight]
    \setlayer[cover]{%
    \starttikzpicture[overlay]
      %% Fill white page with white. Ensure TikZ doesn't relativise our coordinates.
      \fill[fill=white] (0,0) rectangle (210mm,-297mm);

      %% Document title.
      \draw[anchor=west] (5\bodyfontsize,-100mm) node {\bfd\DocTitle};

      %% The MAIN STRIKE insignia.
      \startscope[xshift=150.1mm,yshift=-281mm]
       \startscope[scale=0.060]
        %\draw (0,0) rectangle (1000mm,1000mm);
        \fill (863.200mm,596.700mm) -- (0,235.900mm) -- (0,0);
        \fill (863.200mm,596.700mm) -- (520.600mm,783.400mm) -- (522.600mm,750.400mm);
        %\fill[fill=black] (0,0) rectangle (1000mm,1000mm);
        %\fill[fill=white] (863.200mm,596.700mm) -- (0,235.900mm) -- (0,0);
        %\fill[fill=white] (863.200mm,596.700mm) -- (520.600mm,783.400mm) -- (522.600mm,750.400mm); 
       \stopscope
      \stopscope

      %% Document control information.
      \draw[anchor=south west,align=left] (5\bodyfontsize,-281mm) node[fill=white,draw=black,inner sep=5mm] {{\tt //devever.net/...}
        \doifelse{\DocRevision}{}{}{\\ Revision \DocRevision}%
        %Revision 2021-01-01 ({\tt deadbeef})
      };
    \stoptikzpicture%
    }
    \setupbackgrounds[page][background=cover]
    \popcatcodetable

    %% The empty page.
    \startstandardmakeup[pagestate=start] % Ensure page is numbered
    \stopstandardmakeup
  \endgroup

  %% Copyright Page                                                         {{{1
  %%%%%%%%%%%%%%%%%%%%%%%%%%%%%%%%%%%%%%%%%%%%%%%%%%%%%%%%%%%%%%%%%%%%%%%%%%%%%%
  \begingroup
    \startstandardmakeup[pagestate=start] % Ensure page is numbered
    \vfill
    \xmlfirst{#1}{/s:docctl/s:docinfo/s:dedication}
    \vfill
    Feedback, errata: {\tt hlandau@devever.net}

    \vskip10mm

    {\tt //devever.net/...}\endgraf

    \doifelse{\DocRevision}{}{}{\\ Revision \DocRevision}%
    %Revision 2021-01-01 ({\tt deadbeef})
    \vskip5mm

    System \ConTeXt{}
    \vskip5mm

    © \doifelse{\DocRevision}{}{}{\cldcontext{string.sub([[\DocRevision]],1,4)}} Hugo Landau

    \stopstandardmakeup
  \endgroup

  \title{Table of Contents}
  \placecontent

  %%                                                                        {{{1
  %%%%%%%%%%%%%%%%%%%%%%%%%%%%%%%%%%%%%%%%%%%%%%%%%%%%%%%%%%%%%%%%%%%%%%%%%%%%%%
  
  \xmlfirst{#1}{/s:docctl/s:docinfo/bib:set}
  \xmlfirst{#1}{/s:docbody}
\stopxmlsetups

\startxmlsetups xs:docctl
\xmlflush{#1}%
\stopxmlsetups

\startxmlsetups xs:docinfo
\xmlflush{#1}%
\stopxmlsetups

\startxmlsetups xs:docbody
\xmlflush{#1}
\stopxmlsetups

\startxmlsetups xs:doccontent
\xmlflush{#1}
\stopxmlsetups

\startxmlsetups xs:docproper
\xmlflush{#1}
\stopxmlsetups

\startxmlsetups xs:docmain
\startbodymatter%
\xmlflush{#1}%
\stopbodymatter%
\stopxmlsetups

\startxmlsetups xs:docannex
\startappendices%
\xmlflush{#1}%
\stopappendices%
\stopxmlsetups

\startxmlsetups xs:part
\part{\xmlfirst{#1}{/s:hdr/s:title}}%
\xmlflush{#1}%
\stopxmlsetups

%% DO NOT put \reference outside of a margin note. Placed after the heading command, it causes a single space to appear before the first paragraph. Placed before it, it randomly causes an empty line to appear to appear above the heading. I spent an entire day debugging this.
\startxmlsetups xs:clause
\chapter[id:\xmlatt{#1}{id}]{\xmlfirst{#1}{/s:hdr/s:title}}%
\inleftmargin{\reference[idpfx:\xmlattribute{#1}{.}{id}]{§ }}%
\xmlflush{#1}%
\stopxmlsetups

\startxmlsetups xs:clause2
\section[id:\xmlatt{#1}{id}]{\xmlfirst{#1}{/s:hdr/s:title}}%
\inleftmargin{\reference[idpfx:\xmlattribute{#1}{.}{id}]{§ }}%
\xmlflush{#1}%
\stopxmlsetups

\startxmlsetups xs:clause3
\subsection[id:\xmlatt{#1}{id}]{\xmlfirst{#1}{/s:hdr/s:title}}%
\inleftmargin{\reference[idpfx:\xmlattribute{#1}{.}{id}]{§ }}%
\xmlflush{#1}%
\stopxmlsetups

\startxmlsetups xs:clause4
\subsubsection[id:\xmlatt{#1}{id}]{\xmlfirst{#1}{/s:hdr/s:title}}%
\inleftmargin{\reference[idpfx:\xmlattribute{#1}{.}{id}]{§ }}%
\xmlflush{#1}%
\stopxmlsetups

\startxmlsetups xs:clause5
\subsubsubsection[id:\xmlatt{#1}{id}]{\xmlfirst{#1}{/s:hdr/s:title}}%
\inleftmargin{\reference[idpfx:\xmlattribute{#1}{.}{id}]{§ }}%
\xmlflush{#1}%
\stopxmlsetups

\startxmlsetups xs:clause6
\subsubsubsubsection[id:\xmlatt{#1}{id}]{\xmlfirst{#1}{/s:hdr/s:title}}%
\inleftmargin{\reference[idpfx:\xmlattribute{#1}{.}{id}]{§ }}%
\xmlflush{#1}%
\stopxmlsetups

\startxmlsetups xs:annex
\chapter[id:\xmlatt{#1}{id}]{\xmlfirst{#1}{/s:hdr/s:title}}%
\inleftmargin{\reference[idpfx:\xmlattribute{#1}{.}{id}]{§ }}%
\xmlflush{#1}%
\stopxmlsetups

\startxmlsetups xs:p
\xmlflush{#1}\par%
\stopxmlsetups

\startxmlsetups xs:tp
\hpara{\xmlfirst{#1}{/s:hdr/s:title}}
\xmlflush{#1}%
\stopxmlsetups

\startxmlsetups xs:tt
{\tt \xmlflush{#1}}
\stopxmlsetups

\startxmlsetups xs:code
{\tt \xmlflushspacewise{#1}}
\stopxmlsetups

\startxmlsetups xs:title
\xmlflush{#1}%
\stopxmlsetups

\startxmlsetups xs:description
\xmlflush{#1}%
\stopxmlsetups

\startxmlsetups xs:figure
\placefigure[force][id:\xmlatt{#1}{id}]{\xmlfirst{#1}{/s:hdr/s:title}}{ \xmlflush{#1} }%
\inleftmargin{\reference[idpfx:\xmlatt{#1}{id}]{Figure }}%
\stopxmlsetups

\startxmlsetups xs:table
\placetable[force][id:\xmlatt{#1}{id}]{\xmlfirst{#1}{/s:hdr/s:title}}{ \xmlflush{#1} }%
\inleftmargin{\reference[idpfx:\xmlatt{#1}{id}]{Table }}%
\stopxmlsetups

\startxmlsetups xs:tex
\xmlflushcontext{#1}
\stopxmlsetups

\startxmlsetups xs:pre
{\tt \xmlflushspacewise{#1}}\par
\stopxmlsetups

\startxmlsetups xs:listing
{\tt \xmlflushspacewise{#1}}\par
\stopxmlsetups

\startxmlsetups xs:math2
%\pushcatcodetable%
%\setcatcodetable\ctxcatcodes%
$\xmlflushcontext{#1}$
%\popcatcodetable%
\stopxmlsetups

\startxmlsetups xs:footnote
\footnote{\xmlflush{#1}}
\stopxmlsetups

\startxmlsetups xs:hdr
\stopxmlsetups

\def\iref#1{%
\goto{\ref[title][idpfx:#1]\ref[number][id:#1] (\ref[title][id:#1])}[id:#1]%
}

\def\irefshort#1{%
\goto{\ref[title][idpfx:#1]\ref[number][id:#1]}[id:#1]%
}

\startxmlsetups xs:iref
\csname iref\xmlatt{#1}{fmt}\endcsname{\xmlatt{#1}{iref}}%
\stopxmlsetups

\startxmlsetups xs:ul
  \startitemize%
    \xmlflush{#1}%
  \stopitemize%
\stopxmlsetups

\startxmlsetups xs:ulalso
  \startAlsoList \xmlflush{#1} \stopAlsoList%
\stopxmlsetups

\startxmlsetups xs:ol
  \startitemize[n]%
    \xmlflush{#1}%
  \stopitemize%
\stopxmlsetups

\startxmlsetups xs:li
  \startitem \xmlflush{#1} \stopitem%
\stopxmlsetups

\startxmlsetups xs:termdef
  \masksection{\xmlfirst{#1}{/s:hdr/s:title}}
  \reference[term:\xmlatt{#1}{sym}]{}
  \reference[term:\xmlatt{#1}{sym}s]{}
  \xmlflush{#1}
\stopxmlsetups

\startxmlsetups xs:em
  {\it \xmlflush{#1}}
\stopxmlsetups

\startxmlsetups xs:inlpfx
{\bf \xmlflush{#1}}
\stopxmlsetups

\startxmlsetups xs:procn
  {\sc \xmlflush{#1}}
\stopxmlsetups

\startxmlsetups xs:proword
  {\bf \xmlflush{#1}}
\stopxmlsetups

\startxmlsetups xs:term
  \goto{\color[termgreen]{\xmlflush{#1}}}[term:\xmlatt{#1}{sym}]
\stopxmlsetups

\startxmlsetups xs:table
  \startplacetable[title={\xmlfirst{#1}{/s:hdr/s:title}}]
    \xmlflush{#1}
  \stopplacetable
\stopxmlsetups

\startxmlsetups xs:figtex
\startalignment[middle]%
\xmlcontext{#1}{/s:fig-src}
\stopalignment%
\stopxmlsetups

\def\starttz{\setcatcodetable\ctxcatcodes\starttikzpicture}
\def\stoptz{\stoptikzpicture}

\startxmlsetups xs:figother
  \xmlfirst{#1}{/s:fig-outputs/s:fig-output[@type="application/pdf"]}
\stopxmlsetups

\startxmlsetups xs:fig-output
  \xdef\FigFn{\xmlatt{#1}{href}}
  \externalfigure[\FigPath\FigFn]
\stopxmlsetups

\def\xscite#1{%
\cite[authoryear][#1]%
}

\def\xscited#1{%
\cite[authoryears][#1]%
}

\startxmlsetups xs:cite
\csname xscite\xmlatt{#1}{fmt}\endcsname{\xmlatt{#1}{iref}}%
\stopxmlsetups

\startxmlsetups xs:dict
\xmlflush{#1}%
\stopxmlsetups

\startxmlsetups xs:dice
\startdesc{\xmlfirst{#1}{/s:dick}}%
\xmlfirst{#1}{/s:dicb}%
\stopdesc%
\stopxmlsetups xs:dice

\startxmlsetups xs:dicealsogrouped
\startAlsoGroupedDesc{\xmlfirst{#1}{/s:dick}}%
\xmlfirst{#1}{/s:dicb}%
\stopAlsoGroupedDesc%
\stopxmlsetups

\startxmlsetups xs:dictalso
\bTABLE[frame=off,offset=0mm]
\xmlflush{#1}%
\eTABLE%
\stopxmlsetups

\startxmlsetups xs:dicealso
\bTR\bTD%
\xmlfirst{#1}{/s:dick}%
\eTD\bTD
\xmldoifelseempty{#1}{/s:dicb}{}{\quad — \xmlfirst{#1}{/s:dicb}}%
\eTD\eTR
\stopxmlsetups

\startxmlsetups xs:dick
\xmlflush{#1}%
\stopxmlsetups

\startxmlsetups xs:dicb
\xmlflush{#1}%
\stopxmlsetups

\startxmlsetups xs:bq
\startblockquote%
\xmlflush{#1}%
\stopblockquote%
\stopxmlsetups

\startxmlsetups xs:typeref
\goto{\tt \xmlflush{#1}}[type:#1]%
\stopxmlsetups

\startxmlsetups xs:procref
\goto{\tt \xmlflush{#1}}[proc:#1]%
\stopxmlsetups

\startxmlsetups xs:span
\xmlflush{#1}%
\stopxmlsetups

\startxmlsetups xs:tabular
\bTABLE
\xmlflush{#1}
\eTABLE
\stopxmlsetups

\startxmlsetups xs:tr
\bTR \xmlflush{#1} \eTR
\stopxmlsetups

\startxmlsetups xs:td
\bTD \xmlflush{#1} \eTD
\stopxmlsetups

\startxmlsetups xs:th
\bTD {\bf \xmlflush{#1}} \eTD
\stopxmlsetups

\def\TypeBefore{TYPE }
\startxmlsetups xs:typepage
\refsection{\xmlfirst{#1}{/s:hdr/s:title}}%
%\writetolist[subsection]{}{\hskip10mm {\tt t.} \xmlfirst{#1}{/s:hdr/s:title}}%
\xmlfirst{#1}{/s:hdr/s:description}\par%
\xmlflush{#1}%
\vfill%
\stopxmlsetups

\startxmlsetups xs:tpsynopsys
{\tt \xmlflushspacewise{#1}}\par
%\vskip2mm
\stopxmlsetups

\startxmlsetups xs:tpdescription
\refsubsection{DESCRIPTION}
%\subsubject{DESCRIPTION}
\vskip-1.8mm
\xmlflush{#1}
\stopxmlsetups

\startxmlsetups xs:tpalso
\vbox{%
\refsubsection{SEE ALSO}
%\subsubject{SEE ALSO}
\vskip-1.8mm
\xmlflush{#1}%
}%
\stopxmlsetups

\startxmlsetups xs:procpage
\refsection{\xmlfirst{#1}{/s:hdr/s:title}}
%\writetolist[subsection]{}{\hskip10mm {\tt p.} \xmlfirst{#1}{/s:hdr/s:title}}
\xmlfirst{#1}{/s:hdr/s:description}\par
\xmlflush{#1}
\vfill
\stopxmlsetups

\startxmlsetups xs:ppsynopsys
{\tt \xmlflushspacewise{#1}}\par
%\vskip2mm
\stopxmlsetups

\startxmlsetups xs:ppdescription
\refsubsection{DESCRIPTION}
\vskip-1.8mm
\xmlflush{#1}
\stopxmlsetups

\startxmlsetups xs:argref
{\tt \xmlflush{#1}}%
\stopxmlsetups

\xmldoifelseempty{#1}{/s:dicb}{}{\quad — \xmlfirst{#1}{/s:dicb}}%

\startxmlsetups xs:ppargs
\refsubsection{\xmldoifelseempty{#1}{/s:hdr/s:title}{ARGUMENTS}{\WORDS{\xmlfirst{#1}{/s:hdr/s:title}}}}
\vskip-1.8mm%
\xmlflush{#1}%
\stopxmlsetups

\startxmlsetups xs:pparglist
\startCompactList%
\xmlflush{#1}%
\stopCompactList%
\stopxmlsetups

\startxmlsetups xs:pparg
\startitem {\tt \xmlfirst{#1}{/s:hdr/s:title}} — \xmlflush{#1} \stopitem%
\stopxmlsetups

\startxmlsetups xs:ppword
{\it \xmlflush{#1}}%
\stopxmlsetups

\startxmlsetups xs:ppprecon
\refsubsection{PRECONDITIONS}
\vskip-1.8mm
\xmlflush{#1}
\stopxmlsetups

\startxmlsetups xs:pppostcon
\refsubsection{POSTCONDITIONS}
\vskip-1.8mm
\xmlflush{#1}
\stopxmlsetups

\startxmlsetups xs:conlist
\startCompactList
\xmlflush{#1}
\stopCompactList
\stopxmlsetups

\startxmlsetups xs:unchecked
\startitem {\tt \xmlflush{#1}} (unchecked) \stopitem
\stopxmlsetups

\startxmlsetups xs:ppret
\refsubsection{RETURN VALUE}
\vskip-1.8mm
\xmlflush{#1}
\stopxmlsetups

\startxmlsetups xs:ppworld
\refsubsection{WORLD CONSIDERATIONS}
\vskip-1.8mm
\xmlflush{#1}
\stopxmlsetups

\startxmlsetups xs:ppalso
\refsubsection{SEE ALSO}
\vskip-1.8mm
\xmlflush{#1}%
\stopxmlsetups

\startxmlsetups xs:dualcolumn
\startcolumns[n=2]
\xmlflush{#1}%
\stopcolumns
\stopxmlsetups

\startxmlsetups xs:dedication
\xmlflush{#1}%
\stopxmlsetups

\startxmlsetups xs:regblock
  {\bf \xmlfirst{#1}{/s:title} (\xmlfirst{#1}{/s:mnem})}\endgraf

  \xmlall{#1}{/s:reg}
%       <regblock>
%                <mnem>CTLR</mnem>
%                <title>Controller Registers</title>
%                <reg>
%                  <mnem>CAP</mnem>
%                  <title>Controller Capabilities</title>
%                  <rb-offset>   0h</rb-offset>
%                  <rb-width>64</rb-width>
%                  <rb-access>ro</rb-access>
%                  <field>
%                    <mnem>MQES</mnem>
%                    <bitrange>
%                      <br-lo>0</br-lo>
%                      <br-hi>15</br-hi>
%                    </bitrange>
%                    <rb-access>ro</rb-access>
%                    <rb-title>Maximum Queue Entries Supported</rb-title>
%                    <rb-desc>This field indicates the maximum individual queue size that the controller supports. This value applies to each of the I/O Submission and I/O Completion Queues that host software may create. This is a zeroes-based value. The minimum value is 1, indicating two entries.</rb-desc>
%                  </field>
%
\stopxmlsetups

\startxmlsetups xs:reg
  \hskip.05\textwidth{\bf \xmlfirst{#1}{/s:title} (\xmlfirst{#1}{/s:mnem})}\hfill \xmlfirst{#1}{/s:rb-offset}
  \starttabulate[|rw(.1\textwidth)|p|]
  \xmlall{#1}{/s:field}
  \stoptabulate
\stopxmlsetups

\startxmlsetups xs:field
  \NC \xmlfirst{#1}{/s:bitrange} \NC {\bf \xmlfirst{#1}{/s:mnem}} (\xmlfirst{#1}{/s:rb-title}) \NC\NR
  \NC \NC \xmlfirst{#1}{/s:rb-desc} \NC\NR
\stopxmlsetups

\startxmlsetups xs:bitrange
  \xmldoifelse{#1}{/s:br-lo}{\xmlfirst{#1}{/s:br-lo}:\xmlfirst{#1}{/s:br-hi}}{\xmlfirst{#1}{/s:br-single}}
\stopxmlsetups

\startxmlsetups xbib:set
  \xmlflush{#1}%
  \startluacode
    publications.enhancer.runner(publications.datasets["default"])
  \stopluacode
\stopxmlsetups

\startluacode
  function xml.functions.addbib(t)
    local tag      = xml.attribute(t, ".", "id")
    local category = xml.attribute(t, ".", "category")

    local dict = {}
    for f in xml.collected(t, "/bib:field") do
      local k = xml.attribute(f, ".", "name")
      dict[k] = xml.text(f)
    end
    dict.tag      = tag
    dict.category = category

    publications.loaders.lua("default", { [dict.tag] = dict, })
  end
\stopluacode

\startxmlsetups xbib:entry
  \xmlfunction{#1}{addbib}%
\stopxmlsetups


%% CONFIGURATION                                                           {{{1
%%%%%%%%%%%%%%%%%%%%%%%%%%%%%%%%%%%%%%%%%%%%%%%%%%%%%%%%%%%%%%%%%%%%%%%%%%%%%%%
% TeX remains in the stone ages as far as font fallback support is concerned.
% We have to manually list fallback ranges. This list of ranges is incomplete,
% so this is terribly broken.
\definefontfallback[cjk][name:ArialUnicodeMS][cjkunifiedideographs,cjksymbolsandpunctuation,cjkcompatibility,cjkunifiedideographsextensiona,hiragana,katakana,halfwidthandfullwidthforms,cyrillic]

% Use Helvetica (actually, TeX Gyre Heros).
\definefontsynonym[Sans][heros][fallbacks=cjk]

\loadtypescriptfile[texgyre]

\starttypescriptcollection[myhvmath]
  %\starttypescript[math][myhvmath]
  %  \definefontsynonym[MathRoman][myhvmath@myhvmath-math]
  %  \loadfontgoodies[myhvmath-math]
  %\stoptypescript
  % Helvetica for text _and_ math; but use LM not cursor for mono,
  % since cursor is a bit light.
  \starttypescript[myhvmath]
    \definetypeface[myhvmath][ss][sans][heros][default][rscale=1] %0.9
    \definetypeface[myhvmath][rm][serif][termes][default]
    \definetypeface[myhvmath][tt][mono][modern][default][rscale=1.05]
    %\definetypeface[myhvmath][mm][math][myhvmath][default][rscale=0.9]
    \definetypeface[myhvmath][mm][math][modern][default][rscale=1]
  \stoptypescript
\stoptypescriptcollection

\setupbodyfont[myhvmath,ss,12pt]

% A4.
\definepapersize[main][A4]
\definepapersize[lmain][A4,landscape]
\setuppapersize[main]

% Layout.
\setuplayout[
  width=middle,
  height=middle,
  footer=3\bodyfontsize,
  header=2\bodyfontsize,
  headerdistance=\bodyfontsize,
  topspace=0.5\bodyfontsize,
  leftmargin=5\bodyfontsize,
  rightmargin=5\bodyfontsize
  ]
\definelayout[lmain][
  width=\paperwidth,
  height=\paperheight,
  backspace=0mm
]

\setupstructure[state=start,method=auto]
\setupreferencing[left={},right={}]

% Interaction.
\setupinteraction[state=start,color=,contrastcolor=,style=normal]

% Colours.
\setupcolors[state=start]
\definecolor[grey][r=.7,g=.7,b=.7]
\definecolor[termgreen][r=0,g=.3333,b=0]
\usemodule[tikz]
\usetikzlibrary{positioning,shapes.misc,arrows,arrows.meta,patterns,graphs,fit,calc}

%\setupbackend[export=yes]
%\setupbackend[intent=ISO Uncoated,format=PDF/A-2a]

% Headings.
% Make chapter/section headings have the section number in the margin.
\unexpanded\def\PlaceSection#1#2%
  {\goodbreak
   \vbox
     {\localheadsetup
      \begstrut
      \inleftmargin{ #1}%
      #2}}

\definestructureconversionset[myconv][A,n,n,n,n,n,n][n]
\setupheads[part,chapter,section,subsection,subsubsection,subsubsubsection,subsubsubsubsection][sectionconversionset=myconv]
\setuphead[chapter][sectionsegments=chapter,page=yes,before={\blank[10*big,force]},after=,style=\bfc,command=\PlaceSection]
\setuphead[section][sectionsegments=chapter:subsubsubsubsection,style=\bf,before=,after=,margin=0mm,distance=0mm,textdistance=0mm,command=\PlaceSection,interlinespace=off,alternative=normal]
\definehead[refsection][subsubsubsubsection]
\setuphead[refsection][style=\tfc,page=yes,number=no,continue=no,incrementnumber=yes]
\definehead[refsubsection][subsubsection]
\setuphead[refsubsection][style=\bf,before={\blank[medium]},number=no]

\definesection[hpara-1]
\definehead[hpara][section=hpara-1]
\setuphead[hpara][number=no,incrementnumber=no,alternative=text,textdistance=1em,style=\bf,command={},commandafter={.}]
%\setuphead[hpara][number=no,incrementnumber=no,alternative=text,textdistance=1em,command={},commandafter={.}]

\definehead[masksection][section]

\defineresetset[default][1,0,1][1]
\setuphead[sectionresetset=default]
\setupcaptions[location=top,prefix=yes,way=bychapter,prefixsegments=chapter] %]%,way=bysection,prefixsegments=chapter]

\definedescription[desc][headstyle=bold,style=normal,align=flushleft,alternative=hanging,width=broad,margin=1cm]
\definedescription[AlsoGroupedDesc][headstyle=italic,style=normal,align=flushleft,alternative=top,width=broad,margin=0mm,distance=0mm,before=,after=,inbetween=]

\setupitemgroup[itemize][left=\tfc,inbetween=]

\defineitemgroup[AlsoList]
\setupitemgroup[AlsoList][before=,after=\vskip3mm,inbetween=0mm,symbol=,left=,leftmargin=0mm,leftmargindistance=0mm,margin=0mm,width=0mm,distance=0mm,step=0mm,option=packed,textdistance=0mm,indenting=no]

\defineitemgroup[CompactList]
\setupitemgroup[CompactList][before=,after=,inbetween=0mm,option=packed]

%\definecombinedlist[content][list={part,chapter,section,subsection,subsubsection,subsubsubsection,subsubsubsubsection,refsection}]

% Header text.
\setupheadertexts[]
\setupheadertexts[\setups{text a}][][][\setups{text b}]

\startsetups[text a]
  \framed[width=\textwidth,frame=off,bottomframe=on]{\rlap{Page \pagenumber{} of \totalnumberofpages}
  \hfill
  {\sc \getmarking[chapter]}
  \hfill
  \llap{\sc \getvariable{document}{title}}}
\stopsetups
\startsetups[text b]
  \framed[width=\textwidth,frame=off,bottomframe=on]{\rlap{\sc \getvariable{document}{title}}
  \hfill
  {\sc \getmarking[chapter]}
  \hfill
  \llap{Page \pagenumber{} of \totalnumberofpages}}
\stopsetups

% Paragraph numbering.
\unexpanded\def\NumPara#1%
  {\inleftmargin{\color[grey]{¶#1}}%
  }

\setupwhitespace[small]

\usebtxdefinitions[apa]
\setupbtx[interaction=start,color=black]
\setupbtx[apa:cite:authoryear][etallimit=1,otherstext={ et al.},left={\[},right={\]},inbetween={\btxspace}]
\setupbtx[apa:cite:authoryears][etallimit=1,otherstext={ et al.},inbetween={\btxspace}]

%\setupbtx[default:cite][alternative=authoryear]


%% MAIN                                                                    {{{1
%%%%%%%%%%%%%%%%%%%%%%%%%%%%%%%%%%%%%%%%%%%%%%%%%%%%%%%%%%%%%%%%%%%%%%%%%%%%%%%
\starttext
\xmlprocessfile{dedoc}{\ArgInput}{}
\startbackmatter
\chapter{Bibliography}
\placelistofpublications
\stopbackmatter
\stoptext
